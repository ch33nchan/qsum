\documentclass[11pt,a4paper]{article}
\usepackage[utf8]{inputenc}
\usepackage{amsmath,amssymb,amsthm}
\usepackage{graphicx}
\usepackage{hyperref}
\usepackage{cite}
\usepackage{geometry}
\usepackage{algorithm}
\usepackage{algorithmic}
\usepackage{braket}
\usepackage{tikz}
\usepackage{listings}
\usepackage{xcolor}
\usepackage{subcaption}
\usetikzlibrary{quantikz}

\geometry{margin=1in}

\lstset{
    language=Python,
    basicstyle=\ttfamily\small,
    keywordstyle=\color{blue},
    commentstyle=\color{green},
    stringstyle=\color{red},
    breaklines=true,
    frame=single
}

\title{\textbf{Implementation of Quantum Superposition Gaming in Poker:\\Technical Architecture and Experimental Design}}

\author{Srinivas\\Lossfunk Research Laboratory}
\date{\today}

\begin{document}

\maketitle

\begin{abstract}
This document provides comprehensive implementation details for Quantum Superposition Gaming (QSG) applied to Texas Hold'em poker. We present the complete technical architecture including quantum hand range representations, observation-collapse mechanisms, strategic deception algorithms, and training protocols. Our implementation extends classical poker AI by allowing agents to maintain probabilistic superpositions over hand ranges until strategic revelation events. We detail the mathematical formulations, neural network architectures, training algorithms, and experimental protocols necessary to validate quantum-inspired strategic deception in competitive poker environments.
\end{abstract}

\section{Implementation Overview}

\subsection{System Architecture}

The Quantum Poker Agent (QPA) consists of four primary components:

\begin{enumerate}
\item \textbf{Quantum Hand Range Module}: Maintains superposition states over possible hand holdings
\item \textbf{Observation Engine}: Detects collapse-inducing events from opponent actions
\item \textbf{Strategic Collapse Controller}: Determines optimal collapse timing and target states
\item \textbf{Quantum Policy Network}: Maps superposition states to betting actions
\end{enumerate}

\subsection{Environment Specifications}

We implement QSG within a modified Texas Hold'em environment supporting:

\begin{itemize}
\item Standard 52-card deck with deterministic shuffling for reproducibility
\item No-limit betting structure with configurable stack sizes
\item 2-player heads-up format for simplified analysis
\item Extended action space including quantum-specific operations
\item Enhanced observation space capturing superposition states
\end{itemize}

\section{Quantum Hand Range Representation}

\subsection{Mathematical Foundation}

A quantum hand range is represented as a complex-valued vector over the space of all possible two-card combinations:

\begin{equation}
|\Psi_{\text{range}}\rangle = \sum_{h \in \mathcal{H}} \alpha_h e^{i\phi_h} |h\rangle
\end{equation}

where:
\begin{itemize}
\item $\mathcal{H}$ is the set of all 1326 possible hole card combinations
\item $\alpha_h \in [0,1]$ represents the amplitude for holding hand $h$
\item $\phi_h \in [0, 2\pi]$ encodes strategic phase information
\item Normalization constraint: $\sum_{h} |\alpha_h|^2 = 1$
\end{itemize}

\subsection{Hand Range Evolution}

Hand ranges evolve through unitary transformations based on community cards and strategic decisions:

\begin{align}
|\Psi_{t+1}\rangle &= U_{\text{community}}(c_t) \cdot U_{\text{strategy}}(\theta_t) \cdot |\Psi_t\rangle
\end{align}

where:
\begin{itemize}
\item $U_{\text{community}}(c_t)$ eliminates impossible hands given community card $c_t$
\item $U_{\text{strategy}}(\theta_t)$ applies strategic amplitude modulation
\item Both operators preserve normalization through unitary construction
\end{itemize}

\subsection{Implementation Data Structures}

\begin{lstlisting}
class QuantumHandRange:
    def __init__(self, n_hands=1326):
        self.amplitudes = np.complex128(np.zeros(n_hands))
        self.phases = np.float64(np.zeros(n_hands))
        self.hand_map = self._generate_hand_mapping()
        
    def initialize_uniform_superposition(self):
        """Initialize equal superposition over all hands"""
        self.amplitudes = np.complex128(1.0/np.sqrt(1326))
        self.phases = np.random.uniform(0, 2*np.pi, 1326)
        
    def apply_community_card_constraint(self, community_cards):
        """Remove impossible hands given community cards"""
        valid_mask = self._get_valid_hand_mask(community_cards)
        self.amplitudes[~valid_mask] = 0.0
        self._renormalize()
        
    def get_collapsed_probabilities(self):
        """Return classical probabilities from quantum amplitudes"""
        return np.abs(self.amplitudes)**2
\end{lstlisting}

\section{Observation and Collapse Mechanisms}

\subsection{Observation Events}

We define specific opponent actions that constitute observation events forcing superposition collapse:

\begin{enumerate}
\item \textbf{Aggressive Betting}: Bets $> 2.5 \times$ pot size
\item \textbf{All-in Actions}: Complete stack commitment
\item \textbf{Unusual Sizing}: Bet sizes outside normal ranges
\item \textbf{Temporal Patterns}: Rapid succession of actions
\item \textbf{Showdown}: Mandatory full collapse event
\end{enumerate}

\subsection{Collapse Probability Computation}

The probability of collapse given opponent action $a_{\text{opp}}$ is computed as:

\begin{equation}
P(\text{collapse}|a_{\text{opp}}) = \sigma\left(\mathbf{w}^T \phi(a_{\text{opp}}, |\Psi\rangle, \text{context})\right)
\end{equation}

where:
\begin{itemize}
\item $\phi(\cdot)$ extracts features from action, superposition state, and game context
\item $\mathbf{w}$ are learned collapse sensitivity parameters
\item $\sigma(\cdot)$ is the sigmoid function ensuring $P \in [0,1]$
\end{itemize}

\subsection{Strategic Collapse Selection}

When collapse occurs, the agent strategically selects the target collapsed state:

\begin{equation}
h_{\text{collapse}} = \arg\max_{h} Q_{\text{collapse}}(h | |\Psi\rangle, a_{\text{opp}}, s_t)
\end{equation}

where $Q_{\text{collapse}}$ is a learned value function estimating the strategic benefit of collapsing to specific hands.

\subsection{Collapse Implementation}

\begin{lstlisting}
class CollapseEngine:
    def __init__(self, collapse_net, strategy_net):
        self.collapse_net = collapse_net  # Neural network
        self.strategy_net = strategy_net
        
    def should_collapse(self, opponent_action, quantum_range, context):
        """Determine if opponent action triggers collapse"""
        features = self._extract_collapse_features(
            opponent_action, quantum_range, context
        )
        collapse_prob = torch.sigmoid(self.collapse_net(features))
        return np.random.random() < collapse_prob.item()
        
    def select_collapse_target(self, quantum_range, opponent_action, context):
        """Choose which hand to collapse to"""
        valid_hands = quantum_range.get_nonzero_hands()
        collapse_values = []
        
        for hand in valid_hands:
            hand_features = self._encode_hand_context(
                hand, opponent_action, context
            )
            value = self.strategy_net(hand_features)
            collapse_values.append(value.item())
            
        best_hand_idx = np.argmax(collapse_values)
        return valid_hands[best_hand_idx]
\end{lstlisting}

\section{Neural Network Architecture}

\subsection{Quantum Policy Network}

The quantum policy network processes superposition states and outputs betting actions:

\begin{align}
\text{QPN}(|\Psi\rangle, s_t) &= \text{softmax}(\text{MLP}([\phi_{\text{quantum}}, \phi_{\text{game}}]))
\end{align}

where:
\begin{itemize}
\item $\phi_{\text{quantum}} = [\text{Re}(\alpha_h), \text{Im}(\alpha_h), \phi_h]_{h \in \mathcal{H}}$ encodes superposition
\item $\phi_{\text{game}}$ contains standard poker features (pot size, position, etc.)
\item MLP uses complex-valued activations to preserve quantum information
\end{itemize}

\subsection{Complex-Valued Neural Layers}

We implement custom neural layers for complex-valued computations:

\begin{lstlisting}
class ComplexLinear(nn.Module):
    def __init__(self, in_features, out_features):
        super().__init__()
        self.real_weight = nn.Parameter(torch.randn(out_features, in_features))
        self.imag_weight = nn.Parameter(torch.randn(out_features, in_features))
        self.real_bias = nn.Parameter(torch.randn(out_features))
        self.imag_bias = nn.Parameter(torch.randn(out_features))
        
    def forward(self, x_complex):
        x_real, x_imag = x_complex.real, x_complex.imag
        
        out_real = (torch.mm(x_real, self.real_weight.t()) - 
                   torch.mm(x_imag, self.imag_weight.t()) + 
                   self.real_bias)
        out_imag = (torch.mm(x_real, self.imag_weight.t()) + 
                   torch.mm(x_imag, self.real_weight.t()) + 
                   self.imag_bias)
                   
        return torch.complex(out_real, out_imag)

class QuantumPolicyNetwork(nn.Module):
    def __init__(self, n_hands=1326, hidden_dim=512):
        super().__init__()
        self.quantum_encoder = ComplexLinear(n_hands * 3, hidden_dim)
        self.game_encoder = nn.Linear(64, hidden_dim)  # Standard features
        self.fusion = ComplexLinear(hidden_dim, hidden_dim)
        self.output = nn.Linear(hidden_dim, 4)  # fold, call, raise, quantum_action
        
    def forward(self, quantum_range, game_state):
        # Encode quantum superposition
        quantum_features = torch.cat([
            quantum_range.amplitudes.real,
            quantum_range.amplitudes.imag, 
            quantum_range.phases
        ])
        quantum_encoded = self.quantum_encoder(quantum_features)
        
        # Encode standard game features
        game_encoded = self.game_encoder(game_state)
        game_complex = torch.complex(game_encoded, torch.zeros_like(game_encoded))
        
        # Fuse quantum and classical information
        fused = self.fusion(quantum_encoded + game_complex)
        action_logits = self.output(fused.real)  # Project to real for actions
        
        return torch.softmax(action_logits, dim=-1)
\end{lstlisting}

\subsection{Value Function Architecture}

The quantum value function estimates expected returns from superposition states:

\begin{equation}
V^{\pi}(|\Psi\rangle, s_t) = \mathbb{E}_{h \sim |\Psi\rangle}\left[\sum_{\tau=t}^T \gamma^{\tau-t} r_\tau\right]
\end{equation}

Implementation uses Monte Carlo sampling over possible collapse trajectories:

\begin{lstlisting}
class QuantumValueFunction(nn.Module):
    def __init__(self, n_hands=1326, hidden_dim=256):
        super().__init__()
        self.quantum_processor = ComplexLinear(n_hands * 3, hidden_dim)
        self.game_processor = nn.Linear(64, hidden_dim)
        self.value_head = nn.Linear(hidden_dim * 2, 1)
        
    def forward(self, quantum_range, game_state):
        q_features = self._encode_quantum_state(quantum_range)
        q_processed = self.quantum_processor(q_features).real
        
        g_processed = self.game_processor(game_state)
        
        combined = torch.cat([q_processed, g_processed], dim=-1)
        return self.value_head(combined)
        
    def monte_carlo_evaluation(self, quantum_range, game_state, n_samples=100):
        """Evaluate superposition through sampling"""
        values = []
        for _ in range(n_samples):
            # Sample hand from superposition
            hand = quantum_range.sample_hand()
            # Evaluate classical value with sampled hand
            classical_state = self._create_classical_state(hand, game_state)
            value = self.forward_classical(classical_state)
            values.append(value)
        return torch.mean(torch.stack(values))
\end{lstlisting}

\section{Training Algorithm}

\subsection{Quantum Policy Gradient}

We extend standard policy gradient methods to handle superposition states:

\begin{align}
\nabla_\theta J(\theta) &= \mathbb{E}_{|\Psi\rangle \sim \pi_\theta}\left[\sum_t \nabla_\theta \log \pi_\theta(a_t||\Psi_t\rangle) \cdot A_t\right]
\end{align}

where the advantage function $A_t$ accounts for quantum uncertainty:

\begin{equation}
A_t = Q^{\pi}(|\Psi_t\rangle, a_t) - V^{\pi}(|\Psi_t\rangle)
\end{equation}

\subsection{Collapse Strategy Learning}

The collapse strategy is trained using a separate objective that maximizes expected value improvement:

\begin{align}
\mathcal{L}_{\text{collapse}} &= \mathbb{E}\left[\left(V^{\pi}(h_{\text{collapse}}) - V^{\pi}(|\Psi\rangle)\right)^2\right]
\end{align}

This encourages the agent to collapse to hands that improve its strategic position.

\subsection{Deception Reward Shaping}

We introduce additional reward terms that encourage strategic deception:

\begin{align}
r_{\text{total}} &= r_{\text{game}} + \lambda_1 r_{\text{superposition}} + \lambda_2 r_{\text{deception}} + \lambda_3 r_{\text{collapse}}
\end{align}

where:
\begin{itemize}
\item $r_{\text{superposition}} = \alpha \cdot \text{entropy}(|\Psi\rangle)$ rewards maintaining uncertainty
\item $r_{\text{deception}} = \beta \cdot \text{KL}(\hat{P}_{\text{opp}}||P_{\text{true}})$ rewards misleading opponent beliefs
\item $r_{\text{collapse}} = \gamma \cdot \mathbb{I}[\text{strategic collapse}]$ rewards well-timed revelation
\end{itemize}

\subsection{Training Loop Implementation}

\begin{lstlisting}
class QuantumPokerTrainer:
    def __init__(self, policy_net, value_net, collapse_net):
        self.policy_net = policy_net
        self.value_net = value_net
        self.collapse_net = collapse_net
        self.optimizer = torch.optim.Adam(self.get_all_parameters())
        
    def train_episode(self, opponent_agent):
        episode_data = []
        quantum_range = QuantumHandRange()
        quantum_range.initialize_from_dealt_cards(self.deal_cards())
        
        game_state = self.env.reset()
        done = False
        
        while not done:
            # Get action from quantum policy
            action_probs = self.policy_net(quantum_range, game_state)
            action = self.sample_action(action_probs)
            
            # Execute action and observe
            next_state, reward, done, info = self.env.step(action)
            
            # Check for collapse events
            opponent_action = info.get('opponent_action')
            if self.collapse_engine.should_collapse(opponent_action, quantum_range):
                collapsed_hand = self.collapse_engine.select_collapse_target(
                    quantum_range, opponent_action, game_state
                )
                quantum_range.collapse_to_hand(collapsed_hand)
                
            # Store transition
            episode_data.append({
                'quantum_state': quantum_range.copy(),
                'game_state': game_state,
                'action': action,
                'reward': reward,
                'next_state': next_state
            })
            
            game_state = next_state
            
        # Update networks using collected data
        self.update_networks(episode_data)
        
    def update_networks(self, episode_data):
        policy_loss = self.compute_policy_loss(episode_data)
        value_loss = self.compute_value_loss(episode_data)
        collapse_loss = self.compute_collapse_loss(episode_data)
        
        total_loss = policy_loss + value_loss + collapse_loss
        
        self.optimizer.zero_grad()
        total_loss.backward()
        self.optimizer.step()
\end{lstlisting}

\section{Experimental Protocol}

\subsection{Training Setup}

\textbf{Environment Configuration:}
\begin{itemize}
\item Starting stack: 200 big blinds
\item Blind structure: 1/2 fixed throughout training
\item Hand history logging for detailed analysis
\item Deterministic opponent for initial training
\end{itemize}

\textbf{Network Hyperparameters:}
\begin{itemize}
\item Learning rate: $\alpha = 3 \times 10^{-4}$ with cosine annealing
\item Batch size: 64 episodes
\item Hidden dimensions: 512 for policy, 256 for value
\item Gradient clipping: max norm 0.5
\item Discount factor: $\gamma = 0.99$
\end{itemize}

\textbf{Quantum-Specific Parameters:}
\begin{itemize}
\item Superposition reward weight: $\lambda_1 = 0.1$
\item Deception reward weight: $\lambda_2 = 0.05$
\item Collapse reward weight: $\lambda_3 = 0.02$
\end{itemize}

\subsection{Opponent Models}

We train against progressively sophisticated opponents:

\begin{enumerate}
\item \textbf{Random Agent}: Baseline for sanity checking
\item \textbf{Calling Station}: Always calls, never raises
\item \textbf{Tight-Aggressive Bot}: Classical optimal strategy approximation
\item \textbf{Libratus-Style Agent}: State-of-the-art classical poker AI
\item \textbf{Human Experts}: Professional poker players for validation
\end{enumerate}

\subsection{Evaluation Metrics}

\textbf{Performance Metrics:}
\begin{itemize}
\item Expected value per hand (bb/100)
\item Win rate across different opponent types
\item Variance analysis for risk assessment
\item Convergence speed to optimal policies
\end{itemize}

\textbf{Quantum-Specific Metrics:}
\begin{itemize}
\item Superposition maintenance duration
\item Collapse timing optimality
\item Deception success rate (measured via opponent belief tracking)
\item Quantum strategy diversity (entropy of discovered patterns)
\end{itemize}

\textbf{Emergent Behavior Analysis:}
\begin{itemize}
\item Novel betting patterns not seen in classical poker
\item Strategic collapse timing patterns
\item Quantum bluffing frequency and success
\item Opponent confusion metrics
\end{itemize}

\section{Implementation Challenges and Solutions}

\subsection{Computational Complexity}

\textbf{Challenge}: Maintaining superposition over 1326 possible hands is computationally expensive.

\textbf{Solution}: Implement sparse superposition representation focusing on strategically relevant hands:

\begin{lstlisting}
class SparseQuantumRange:
    def __init__(self, sparsity_threshold=1e-6):
        self.active_hands = {}  # hand_id -> amplitude
        self.threshold = sparsity_threshold
        
    def prune_negligible_amplitudes(self):
        """Remove hands with negligible probability"""
        to_remove = [h for h, amp in self.active_hands.items() 
                    if abs(amp)**2 < self.threshold]
        for h in to_remove:
            del self.active_hands[h]
            
    def adaptive_sparsity(self, target_size=100):
        """Keep only top-k most probable hands"""
        if len(self.active_hands) <= target_size:
            return
        sorted_hands = sorted(self.active_hands.items(), 
                            key=lambda x: abs(x[1])**2, reverse=True)
        self.active_hands = dict(sorted_hands[:target_size])
        self._renormalize()
\end{lstlisting}

\subsection{Training Stability}

\textbf{Challenge}: Complex-valued networks can exhibit training instabilities.

\textbf{Solution}: Implement specialized initialization and regularization:

\begin{lstlisting}
def initialize_complex_weights(layer):
    """Initialize complex weights for stable training"""
    if isinstance(layer, ComplexLinear):
        # Xavier initialization for complex weights
        fan_in = layer.real_weight.size(1)
        std = np.sqrt(2.0 / fan_in)
        layer.real_weight.data.normal_(0, std)
        layer.imag_weight.data.normal_(0, std)
        layer.real_bias.data.zero_()
        layer.imag_bias.data.zero_()

def complex_gradient_clipping(model, max_norm=1.0):
    """Clip gradients for complex-valued parameters"""
    total_norm = 0
    for p in model.parameters():
        if p.grad is not None:
            param_norm = p.grad.data.norm(2)
            total_norm += param_norm.item() ** 2
    total_norm = total_norm ** (1. / 2)
    
    clip_coef = max_norm / (total_norm + 1e-6)
    if clip_coef < 1:
        for p in model.parameters():
            if p.grad is not None:
                p.grad.data.mul_(clip_coef)
\end{lstlisting}

\subsection{Opponent Belief Modeling}

\textbf{Challenge}: Measuring deception success requires accurate opponent belief tracking.

\textbf{Solution}: Implement Bayesian opponent model with uncertainty quantification:

\begin{lstlisting}
class OpponentBeliefTracker:
    def __init__(self, n_hands=1326):
        self.belief_history = []
        self.action_history = []
        self.hand_prior = np.ones(n_hands) / n_hands
        
    def update_belief(self, opponent_action, community_cards):
        """Update belief about opponent hand range"""
        # Bayes update based on action
        likelihood = self._compute_action_likelihood(
            opponent_action, community_cards
        )
        
        posterior = self.hand_prior * likelihood
        posterior /= np.sum(posterior)
        
        self.belief_history.append(posterior.copy())
        self.action_history.append(opponent_action)
        self.hand_prior = posterior
        
    def compute_deception_metric(self, true_quantum_range):
        """Measure how much opponent belief differs from reality"""
        if not self.belief_history:
            return 0.0
            
        true_probs = true_quantum_range.get_collapsed_probabilities()
        believed_probs = self.belief_history[-1]
        
        # KL divergence as deception measure
        kl_div = np.sum(true_probs * np.log(
            (true_probs + 1e-8) / (believed_probs + 1e-8)
        ))
        return kl_div
\end{lstlisting}

\section{Expected Results and Analysis}

\subsection{Predicted Emergent Behaviors}

\textbf{Quantum Bluffing Patterns:}
\begin{itemize}
\item Maintaining superposition over strong and weak hands simultaneously
\item Strategic collapse to unexpected hand strengths
\item Temporal deception through delayed revelation
\end{itemize}

\textbf{Novel Betting Strategies:}
\begin{itemize}
\item Bet sizes that reflect superposition uncertainty
\item Action sequences impossible in classical poker
\item Meta-strategic collapse timing
\end{itemize}

\textbf{Opponent Exploitation:}
\begin{itemize}
\item Learning opponent collapse triggers
\item Exploiting opponent belief models
\item Creating cognitive overload through quantum complexity
\end{itemize}

\subsection{Performance Predictions}

Based on theoretical analysis, we expect:
\begin{itemize}
\item 15-25\% improvement in expected value against classical opponents
\item Novel strategic patterns not present in existing poker literature
\item Emergent quantum strategies that transfer to other game domains
\item Demonstration of genuine strategic advantages from superposition maintenance
\end{itemize}

\section{Conclusion}

This implementation guide provides a complete technical roadmap for creating the first quantum superposition gaming agent applied to poker. The combination of rigorous mathematical foundations, practical implementation details, and comprehensive experimental protocols creates a framework for validating quantum-inspired strategic deception in competitive environments.

The poker domain serves as an ideal testbed for QSG principles, offering natural hidden information structures, clear observation mechanics, and measurable strategic outcomes. Success in this implementation would establish the viability of quantum superposition concepts for broader game-theoretic applications while opening entirely new research directions in strategic AI.

The technical challenges outlined here, while significant, are addressable through the proposed solutions and represent genuine contributions to both reinforcement learning and game theory. The expected emergent behaviors would constitute novel strategic patterns previously impossible in classical gaming frameworks.

\bibliographystyle{plain}
\bibliography{references}

\end{document}